% Options for packages loaded elsewhere
\PassOptionsToPackage{unicode,linktoc=all}{hyperref}
\PassOptionsToPackage{hyphens}{url}
\PassOptionsToPackage{dvipsnames,svgnames,x11names}{xcolor}
%
\documentclass[
  12pt,
  a4paperpaper,
  DIV=11,
  numbers=noendperiod]{scrartcl}

\usepackage{amsmath,amssymb}
\usepackage{iftex}
\ifPDFTeX
  \usepackage[T1]{fontenc}
  \usepackage[utf8]{inputenc}
  \usepackage{textcomp} % provide euro and other symbols
\else % if luatex or xetex
  \usepackage{unicode-math}
  \defaultfontfeatures{Scale=MatchLowercase}
  \defaultfontfeatures[\rmfamily]{Ligatures=TeX,Scale=1}
\fi
\usepackage{lmodern}
\ifPDFTeX\else  
    % xetex/luatex font selection
    \setmainfont[]{Gentium Plus}
    \setsansfont[]{Gentium Plus}
  \setmathfont[]{Gentium Plus}
\fi
% Use upquote if available, for straight quotes in verbatim environments
\IfFileExists{upquote.sty}{\usepackage{upquote}}{}
\IfFileExists{microtype.sty}{% use microtype if available
  \usepackage[]{microtype}
  \UseMicrotypeSet[protrusion]{basicmath} % disable protrusion for tt fonts
}{}
\makeatletter
\@ifundefined{KOMAClassName}{% if non-KOMA class
  \IfFileExists{parskip.sty}{%
    \usepackage{parskip}
  }{% else
    \setlength{\parindent}{0pt}
    \setlength{\parskip}{6pt plus 2pt minus 1pt}}
}{% if KOMA class
  \KOMAoptions{parskip=half}}
\makeatother
\usepackage{xcolor}
\usepackage[top=20mm,bottom=20mm,left=20mm,right=20mm,bindingoffset=0mm]{geometry}
\setlength{\emergencystretch}{3em} % prevent overfull lines
\setcounter{secnumdepth}{5}
% Make \paragraph and \subparagraph free-standing
\makeatletter
\ifx\paragraph\undefined\else
  \let\oldparagraph\paragraph
  \renewcommand{\paragraph}{
    \@ifstar
      \xxxParagraphStar
      \xxxParagraphNoStar
  }
  \newcommand{\xxxParagraphStar}[1]{\oldparagraph*{#1}\mbox{}}
  \newcommand{\xxxParagraphNoStar}[1]{\oldparagraph{#1}\mbox{}}
\fi
\ifx\subparagraph\undefined\else
  \let\oldsubparagraph\subparagraph
  \renewcommand{\subparagraph}{
    \@ifstar
      \xxxSubParagraphStar
      \xxxSubParagraphNoStar
  }
  \newcommand{\xxxSubParagraphStar}[1]{\oldsubparagraph*{#1}\mbox{}}
  \newcommand{\xxxSubParagraphNoStar}[1]{\oldsubparagraph{#1}\mbox{}}
\fi
\makeatother


\providecommand{\tightlist}{%
  \setlength{\itemsep}{0pt}\setlength{\parskip}{0pt}}\usepackage{longtable,booktabs,array}
\usepackage{calc} % for calculating minipage widths
% Correct order of tables after \paragraph or \subparagraph
\usepackage{etoolbox}
\makeatletter
\patchcmd\longtable{\par}{\if@noskipsec\mbox{}\fi\par}{}{}
\makeatother
% Allow footnotes in longtable head/foot
\IfFileExists{footnotehyper.sty}{\usepackage{footnotehyper}}{\usepackage{footnote}}
\makesavenoteenv{longtable}
\usepackage{graphicx}
\makeatletter
\newsavebox\pandoc@box
\newcommand*\pandocbounded[1]{% scales image to fit in text height/width
  \sbox\pandoc@box{#1}%
  \Gscale@div\@tempa{\textheight}{\dimexpr\ht\pandoc@box+\dp\pandoc@box\relax}%
  \Gscale@div\@tempb{\linewidth}{\wd\pandoc@box}%
  \ifdim\@tempb\p@<\@tempa\p@\let\@tempa\@tempb\fi% select the smaller of both
  \ifdim\@tempa\p@<\p@\scalebox{\@tempa}{\usebox\pandoc@box}%
  \else\usebox{\pandoc@box}%
  \fi%
}
% Set default figure placement to htbp
\def\fps@figure{htbp}
\makeatother
% definitions for citeproc citations
\NewDocumentCommand\citeproctext{}{}
\NewDocumentCommand\citeproc{mm}{%
  \begingroup\def\citeproctext{#2}\cite{#1}\endgroup}
\makeatletter
 % allow citations to break across lines
 \let\@cite@ofmt\@firstofone
 % avoid brackets around text for \cite:
 \def\@biblabel#1{}
 \def\@cite#1#2{{#1\if@tempswa , #2\fi}}
\makeatother
\newlength{\cslhangindent}
\setlength{\cslhangindent}{1.5em}
\newlength{\csllabelwidth}
\setlength{\csllabelwidth}{3em}
\newenvironment{CSLReferences}[2] % #1 hanging-indent, #2 entry-spacing
 {\begin{list}{}{%
  \setlength{\itemindent}{0pt}
  \setlength{\leftmargin}{0pt}
  \setlength{\parsep}{0pt}
  % turn on hanging indent if param 1 is 1
  \ifodd #1
   \setlength{\leftmargin}{\cslhangindent}
   \setlength{\itemindent}{-1\cslhangindent}
  \fi
  % set entry spacing
  \setlength{\itemsep}{#2\baselineskip}}}
 {\end{list}}
\usepackage{calc}
\newcommand{\CSLBlock}[1]{\hfill\break\parbox[t]{\linewidth}{\strut\ignorespaces#1\strut}}
\newcommand{\CSLLeftMargin}[1]{\parbox[t]{\csllabelwidth}{\strut#1\strut}}
\newcommand{\CSLRightInline}[1]{\parbox[t]{\linewidth - \csllabelwidth}{\strut#1\strut}}
\newcommand{\CSLIndent}[1]{\hspace{\cslhangindent}#1}

\usepackage{mhchem}
\usepackage{float}
\usepackage{gensymb}
\usepackage{unicode-math}
\usepackage{setspace}
\usepackage{array}
\linespread{1.0}
\newcolumntype{L}[1]{>{\raggedright\arraybackslash}p{#1}}
\KOMAoption{captions}{tableheading}
\makeatletter
\@ifpackageloaded{caption}{}{\usepackage{caption}}
\AtBeginDocument{%
\ifdefined\contentsname
  \renewcommand*\contentsname{Table of contents}
\else
  \newcommand\contentsname{Table of contents}
\fi
\ifdefined\listfigurename
  \renewcommand*\listfigurename{List of Figures}
\else
  \newcommand\listfigurename{List of Figures}
\fi
\ifdefined\listtablename
  \renewcommand*\listtablename{List of Tables}
\else
  \newcommand\listtablename{List of Tables}
\fi
\ifdefined\figurename
  \renewcommand*\figurename{Figure}
\else
  \newcommand\figurename{Figure}
\fi
\ifdefined\tablename
  \renewcommand*\tablename{Table}
\else
  \newcommand\tablename{Table}
\fi
}
\@ifpackageloaded{float}{}{\usepackage{float}}
\floatstyle{ruled}
\@ifundefined{c@chapter}{\newfloat{codelisting}{h}{lop}}{\newfloat{codelisting}{h}{lop}[chapter]}
\floatname{codelisting}{Listing}
\newcommand*\listoflistings{\listof{codelisting}{List of Listings}}
\makeatother
\makeatletter
\usepackage{pdflscape}
\makeatother
\makeatletter
\makeatother
\makeatletter
\@ifpackageloaded{caption}{}{\usepackage{caption}}
\@ifpackageloaded{subcaption}{}{\usepackage{subcaption}}
\makeatother

\ifLuaTeX
\usepackage[bidi=basic]{babel}
\else
\usepackage[bidi=default]{babel}
\fi
\babelprovide[main,import]{american}
\ifPDFTeX
\else
\babelfont{rm}[]{Gentium Plus}
\fi
% get rid of language-specific shorthands (see #6817):
\let\LanguageShortHands\languageshorthands
\def\languageshorthands#1{}
\ifLuaTeX
  \usepackage[english]{selnolig} % disable illegal ligatures
\fi
\usepackage{bookmark}

\IfFileExists{xurl.sty}{\usepackage{xurl}}{} % add URL line breaks if available
\urlstyle{same} % disable monospaced font for URLs
\hypersetup{
  pdftitle={Modeling of packed-bed reactor for the water-gas shift reaction},
  pdflang={en-US},
  colorlinks=true,
  linkcolor={blue},
  filecolor={Maroon},
  citecolor={Blue},
  urlcolor={Blue},
  pdfcreator={LaTeX via pandoc}}


\title{Modeling of packed-bed reactor for the water-gas shift reaction}
\usepackage{etoolbox}
\makeatletter
\providecommand{\subtitle}[1]{% add subtitle to \maketitle
  \apptocmd{\@title}{\par {\large #1 \par}}{}{}
}
\makeatother
\subtitle{Group Project: chemical reaction engineering S1 2025}
\author{}
\date{}

\begin{document}
\maketitle


\section{Introduction}\label{introduction}

Hydrogen is rapidly emerging as a key player in the global shift toward
clean energy.

The primary production route is through methane reforming, which
includes three main processes: steam reforming, autothermal reforming,
and partial oxidation. Steam reforming is endothermic (requires heat),
while partial oxidation is exothermic (releases heat). These reactions
generate hydrogen along with carbon monoxide (CO), which must be removed
for applications such as fuel cells, where CO levels must be below 10
ppm due to its poisoning effect on the electrocatalyst.

The water-gas shift (WGS) reaction is a key step in hydrogen production
processes, particularly for lowering carbon monoxide (CO) concentrations
in reformate streams to levels suitable for fuel cell applications. Due
to its exothermic nature and equilibrium-limited conversion, the WGS
reaction is typically conducted in two stages: a high-temperature (HT)
stage using Fe-based catalysts and a low-temperature (LT) stage using
Cu-based catalysts. Among low-temperature catalysts, commercial
CuO/ZnO/Al₂O₃ formulations have demonstrated superior performance in
terms of CO conversion and operational stability.

To effectively utilize this catalyst in practical systems, there is a
growing need for accurate modeling of packed-bed reactors used for the
WGS reaction. These reactors are increasingly considered for compact,
small-scale hydrogen generation units that can integrate with membrane
reactors for producing high-purity H₂. Existing studies have provided
kinetic parameters for the CuO/ZnO/Al₂O₃ catalyst based on
Langmuir-Hinshelwood and Redox models, validated through experimental
data over a range of temperatures (150--300 °C). Accurate reactor
modeling is essential for designing and optimizing such systems,
especially when targeting high CO conversion and integration with
downstream purification units.

\section{Problem description}\label{problem-description}

In this project, your team has been assigned the task of developing a
reaction engineering model for a fixed-bed packed-bed reactor (PBR)
employing a CuO/ZnO/Al₂O₃ catalyst for the low-temperature WGS reaction.
The model will use previously reported kinetic expressions to simulate
reactor behavior and will be validated against available experimental
data. In addition, the study will perform systematic parametric analyses
to investigate the effects of key process variables---including
temperature, pressure, and feed gas composition---on reactor
performance. The goal is to develop a predictive model that captures the
core reaction behavior and supports the design and operation of WGS
reactors in hydrogen production systems.

The base references for this project are (Manrique et al.
(\citeproc{ref-manrique2012ijcre}{2012}), Zhou et al.
(\citeproc{ref-zhou2023joec}{2023}))

\subsection{Reaction Engineering Aspects of the Water-Gas Shift
Reaction}\label{reaction-engineering-aspects-of-the-water-gas-shift-reaction}

Prepare a comprehensive review (4-5 pages) on the reaction engineering
fundamentals of the Water-Gas Shift (WGS) reaction. Cover key areas such
as the industrial relevance of the WGS reaction, its role in hydrogen
production, the catalytic systems used (including high-temperature
iron-based and low-temperature copper-based catalysts), and mechanistic
insights such as Langmuir-Hinshelwood and Redox pathways. Discuss
reactor configurations (e.g., adiabatic vs.~isothermal packed beds,
membrane reactors), typical operating conditions, and challenges such as
CO poisoning in downstream fuel cells. Highlight recent advances in
catalyst development and reactor miniaturization for small-scale H₂
generation. Use the provided reference (Manrique et al.
(\citeproc{ref-manrique2012ijcre}{2012}), Zhou et al.
(\citeproc{ref-zhou2023joec}{2023})) as a starting point to support
discussions on kinetic models and the rationale behind the choice of
catalyst and reactor type for this study.

\subsection{Modeling of a Packed-Bed Reactor for the WGS
Reaction}\label{modeling-of-a-packed-bed-reactor-for-the-wgs-reaction}

Develop a one-dimensional isothermal reaction engineering model of a
packed-bed reactor (PBR) for the low-temperature WGS reaction using the
CuO/ZnO/Al₂O₃ catalyst. Incorporate kinetic expressions as reported by
Mendes et al. (\citeproc{ref-mendes2010iecr}{2010}) applying the
Langmuir-Hinshelwood model at temperatures below 215\,°C and a Redox
mechanism above that threshold. Use the modeling approach described by
Manrique et al. (\citeproc{ref-manrique2012ijcre}{2012}) to formulate
mass balance equations assuming ideal packed bed reactor. Simulate
species concentration profiles and CO conversion using appropriate
computational tool (such as python or similar tools). Validate model
predictions with experimental data from literature. State all
assumptions clearly, justify them with literature, and tabulate physical
and kinetic parameters used.

\subsection{Parametric Analysis of Process
Variables}\label{parametric-analysis-of-process-variables}

Conduct a detailed parametric study to evaluate the influence of key
process variables such as temperature, pressure, feed composition (CO
and H₂O content), and flow rate on reactor performance. Use conditions
and data from Manrique et al. (\citeproc{ref-manrique2012ijcre}{2012}).
Use your engineering judgment, literature sources, etc. for any missing
data. Use the validated reactor model to simulate changes in CO
conversion and hydrogen yield under different operating scenarios.
Analyze trends in the results with reference to thermodynamic
limitations and kinetic control. Compare trends with those reported in
Manrique et al. (\citeproc{ref-manrique2012ijcre}{2012}), including how
conversion responds to increased space time or pressure. Conclude this
section with a discussion on optimal operating conditions for maximizing
hydrogen production. Include brief notes on implications for reactor
control and safety (e.g., sensitivity to temperature changes and
implications for catalyst stability).

The project is deliberately set to be open‐ended. You are expected to do
some self‐directed study of material outside of what has been covered in
the unit.

\section{Report}\label{report}

Prepare a report consisting of the following:

\begin{enumerate}
\def\labelenumi{\arabic{enumi}.}
\item
  Literature review on reaction engineering aspects of the Water-Gas
  Shift (WGS) reaction.

  Present a comprehensive note (5--6 pages) on the WGS reaction
  including catalysis, reaction mechanisms, temperature regimes, reactor
  types, and recent advances.
\item
  Packed-bed reactor modeling and simulation.

  Develop a reaction engineering model for a WGS packed-bed reactor,
  simulate performance, and validate with available experimental data.
\item
  Parametric studies and analysis.

  Study the effect of key variables like temperature, pressure, and feed
  composition on CO conversion and H₂ production.
\item
  Critical review.

  Provide a reflective analysis of your modeling work: What was
  reliable? What assumptions were made? How can the model be improved?
\end{enumerate}

\subsection{Marking}\label{marking}

\begin{longtable}[]{@{}ll@{}}
\toprule\noalign{}
Description & Marks \\
\midrule\noalign{}
\endhead
\bottomrule\noalign{}
\endlastfoot
Short note & 20 \\
Reaction engineering aspects of WGS & \\
Modeling and Simulation & 60 \\
- Reactor model development and validation & \emph{40} \\
- Parametric analysis & \emph{20} \\
Critical Review & 10 \\
Report Presentation & 10 \\
Total & 100 \\
\end{longtable}

See detailed rubric in Section~\ref{sec-rubric} for marking key.

\subsection{Report format}\label{report-format}

The following guidelines are presented to ensure uniformity, clarity,
and professionalism in your report submission.

\subsubsection{Cover page}\label{cover-page}

The cover page should have following information

\textbf{Project Title:}\\
Modeling and Simulation of a Packed-Bed Reactor for the Water-Gas Shift
Reaction

\textbf{Submitted by:}\\
Student Names and IDs:\\
- Student 1 (ID: )\\
- Student 2 (ID: )\\
- Student 3 (ID: )\\
- Student 4 (ID: )

\textbf{Date of Submission:}\\
{[}DD Month YYYY{]}

\textbf{Peer Contribution}

\begin{longtable}[]{@{}ll@{}}
\toprule\noalign{}
Team Member & Overall Contribution (\%) \\
\midrule\noalign{}
\endhead
\bottomrule\noalign{}
\endlastfoot
Member 1 & \\
Member 2 & \\
Member 3 & \\
Member 4 & \\
\end{longtable}

\subsubsection{General requirements}\label{general-requirements}

\begin{enumerate}
\def\labelenumi{\arabic{enumi}.}
\tightlist
\item
  Maximum Length:

  \begin{itemize}
  \tightlist
  \item
    \textbf{30 pages} total (excluding references, appendices, and
    nomenclature).
  \item
    Pages exceeding this limit will carry a 10\% penalty.
  \end{itemize}
\item
  Font \& Text Formatting:

  \begin{itemize}
  \tightlist
  \item
    Font: Standard professional font either sans or sans serif
  \item
    Font Size: Main text:11 pt minimum; Captions, footnotes, references:
    9--10 pt minimum
  \item
    Line Spacing: 1.15 minimum
  \item
    Text Alignment: Justified
  \item
    Paragraph Spacing: 6 pt after each paragraph
  \item
    Section Headings: Use numbered sections (e.g., 3.2 Catalyst types
    and activity)
  \item
    Subheadings: Use consistent formatting
  \end{itemize}
\item
  Page Layout:

  \begin{itemize}
  \tightlist
  \item
    Paper Size: A4
  \item
    Margins: minimum 2 cm on all sides
  \item
    Header/Footer: May be used for page numbers and project title
  \item
    Page Numbers: Bottom-center or bottom-right, starting after the
    cover page
  \end{itemize}
\end{enumerate}

\subsubsection{Figures, tables, and
equations}\label{figures-tables-and-equations}

\begin{itemize}
\tightlist
\item
  Figures/Tables:

  \begin{itemize}
  \tightlist
  \item
    Must be numbered (e.g., \emph{Figure 3.2}, \emph{Table 5.1})
  \item
    Caption placed below figures, above tables
  \item
    Cite in text (e.g., ``as shown in Figure 3.2'')
  \end{itemize}
\item
  Equations:

  \begin{itemize}
  \tightlist
  \item
    Center aligned
  \item
    Numbered on the right (e.g., (1), (2))
  \item
    Use consistent symbols and define them in the nomenclature section
  \end{itemize}
\end{itemize}

\subsubsection{References}\label{references}

\begin{itemize}
\tightlist
\item
  Citation Style: APA6 or Chicago (consistent throughout)
\item
  All references must be cited in-text
\item
  Include journal articles, books, and relevant technical standards
\item
  Suggested minimum: 10 quality references
\end{itemize}

\subsubsection{Technical writing}\label{technical-writing}

\begin{itemize}
\tightlist
\item
  Avoid informal language
\item
  Use passive or formal voice (e.g., ``The reactor was modeled
  using\ldots{}'')
\item
  Define all acronyms upon first use
\item
  Be concise and clear -- avoid redundant explanations
\item
  Avoid large blocks of text -- use figures, tables, bullet points where
  suitable
\end{itemize}

\subsection{Report structure}\label{report-structure}

The following structure is recommended for your report:

\begin{verbatim}
Cover page
1.0 Executive Summary
2.0 Introduction to the Water-Gas Shift Reaction
3.0 Reaction Engineering Aspects
    3.1 Reaction mechanism and thermodynamics
    3.2 Catalyst types and activity
    3.3 Operating regimes (HT/LT WGS)
4.0 Reactor Model Development
    4.1 Reactor configuration and assumptions
    4.2 Kinetic model
    4.3 Governing equations
    4.4 Numerical methods and solution approach
5.0 Model Validation
    5.1 Comparison with experimental data
6.0 Parametric Studies
    6.1 Effect of temperature
    6.2 Effect of pressure
    6.3 Effect of feed composition
7.0 Critical Review
8.0 Conclusions and Recommendations
9.0 References
10.0 Appendices
\end{verbatim}

Appendices should include supporting material that is too detailed for
the main body but still essential for completeness, transparency, or
reproducibility. Here's what you can include in the appendices for your
WGS reactor modeling project:

\begin{itemize}
\item
  Detailed calculations

  \begin{itemize}
  \tightlist
  \item
    Step-by-step derivations not shown in the main report
  \end{itemize}
\item
  Kinetic parameters and data tables
\item
  Complete list of values used (e.g., from Mendes et al., 2010a)
\item
  Experimental conditions, catalyst properties, etc.
\item
  Python code

  \begin{itemize}
  \tightlist
  \item
    Scripts used for simulation or solving ODEs
  \item
    Only include key parts if the full code is long
  \end{itemize}
\item
  Format guidelines

  \begin{itemize}
  \tightlist
  \item
    Use section numbers (e.g., Appendix A, Appendix B)
  \item
    Reference them in the main text (e.g., ``see Appendix B for full
    kinetic parameters'')
  \item
    Ensure readability and proper formatting even for raw data or code
  \end{itemize}
\end{itemize}

\subsection{Submission Checklist}\label{submission-checklist}

\begin{itemize}
\tightlist
\item[$\square$]
  Cover page includes project title, student names, group number, and
  date\\
\item[$\square$]
  All required sections are included\\
\item[$\square$]
  Page limit not exceeded\\
\item[$\square$]
  References formatted and cited properly\\
\item[$\square$]
  Figures and tables clearly labeled\\
\item[$\square$]
  Equation numbering is consistent\\
\item[$\square$]
  Proofread for grammar and clarity
\end{itemize}

\section{Submission}\label{submission}

\textbf{Bentley Students}:

The project is conducted in a group of four. You are free to choose your
group. Please notify the instructors of your groups as soon as you form
them. If you cannot find a group, please get in touch with your
instructor at the earliest.

\textbf{Miri Students}:

Your project group will be same as your assigned lab group.

\textbf{Submission instructions}:

You will need to submit all the files created electronically on
blackboard There should be one submission per group. Please follow the
instructions given below carefully for preparing the files for
submission. Failure to follow these instructions may result in us not
being able to assess the files.

You will be uploading two files.

\begin{enumerate}
\def\labelenumi{\arabic{enumi}.}
\item
  Report (pdf file containing the report). You need to name the file as
  STUDENTID\_CHEN3010\_project\_report.pdf (or )
  STUDENTID\_CHEN5040\_project\_report.pdf (Replace STUDENTID with your
  Student ID; Miri students replace STUDENTID with Group number). You
  need to make only one submission per group.
\item
  Create a zip file named
  STUDENTID\_CHEN3010\_project\_Supporting\_files.zip. The zip file
  should contain a) All supporting files for design, modeling, and
  simulation activities (excel, python, \ldots) presented in the PDF
  report file. You may upload the supporting file to a cloud storage of
  your preference and share a link with us.
\end{enumerate}

\newpage{}

\begin{landscape}

\section{Marking rubric}\label{sec-rubric}

\begin{longtable}[]{@{}
  >{\raggedright\arraybackslash}p{(\linewidth - 10\tabcolsep) * \real{0.1667}}
  >{\raggedright\arraybackslash}p{(\linewidth - 10\tabcolsep) * \real{0.1667}}
  >{\raggedright\arraybackslash}p{(\linewidth - 10\tabcolsep) * \real{0.1667}}
  >{\raggedright\arraybackslash}p{(\linewidth - 10\tabcolsep) * \real{0.1667}}
  >{\raggedright\arraybackslash}p{(\linewidth - 10\tabcolsep) * \real{0.1667}}
  >{\raggedright\arraybackslash}p{(\linewidth - 10\tabcolsep) * \real{0.1667}}@{}}
\toprule\noalign{}
\begin{minipage}[b]{\linewidth}\raggedright
Marking Criteria
\end{minipage} & \begin{minipage}[b]{\linewidth}\raggedright
Unsatisfactory (Fail) {[}0 -- 49\%{]}
\end{minipage} & \begin{minipage}[b]{\linewidth}\raggedright
Satisfactory (Pass) {[}50 -- 59\%{]}
\end{minipage} & \begin{minipage}[b]{\linewidth}\raggedright
Competent (Credit) {[}60 -- 69\%{]}
\end{minipage} & \begin{minipage}[b]{\linewidth}\raggedright
Very Competent (Distinction) {[}70 -- 79\%{]}
\end{minipage} & \begin{minipage}[b]{\linewidth}\raggedright
Excellent (High Distinction) {[}80 -- 100\%{]}
\end{minipage} \\
\midrule\noalign{}
\endhead
\bottomrule\noalign{}
\endlastfoot
& & & & & \\
1. Reaction Engineering Aspects of the WGS Reaction (20 marks) & No or
incomplete discussion, lacking key concepts. {[}0 -- 9.8{]} & Basic
discussion on WGS reaction, with minimal detail. {[}10.0 -- 11.8{]} &
Covers mechanism, catalysts, and process conditions with some technical
depth. {[}12.0 -- 13.8{]} & Covers all core aspects in detail with
comparisons and literature support. {[}14.0 -- 15.8{]} & Thorough,
well-structured, and critically informed note with detailed references
and examples. {[}16.0 -- 20.0{]} \\
& & & & & \\
2. Packed-Bed Reactor Modeling (40 marks) & No clear modeling approach
or incorrect equations used. {[}0 -- 19.6{]} & Basic plug-flow model
with limited explanation and minimal validation. {[}20.0 -- 23.6{]} &
Clear model development with reasonable assumptions and comparison to
some data. {[}24.0 -- 27.6{]} & Detailed modeling using literature
kinetics, supported with validation and analysis. {[}28.0 -- 31.6{]} &
Complete model, well-validated, technically sound, and thoroughly
justified using reference data. {[}32.0 -- 40.0{]} \\
& & & & & \\
3. Parametric Analysis of Process Variables (20 marks) & No analysis or
disconnected from the model. {[}0 -- 9.8{]} & General trends described,
limited linkage to model outputs. {[}10.0 -- 11.8{]} & Effect of key
parameters like temperature/flow rate analyzed using model. {[}12.0 --
13.8{]} & Parametric trends explained with supporting simulations and
discussion. {[}14.0 -- 15.8{]} & Deep and insightful analysis with clear
linkage to design and optimization decisions. {[}16.0 -- 20.0{]} \\
& & & & & \\
4. Critical Review (10 marks) & No reflection or irrelevant content.
{[}0 -- 4.9{]} & Basic reflection on results and limitations. {[}5.0 --
5.9{]} & Highlights strengths and weaknesses with some suggestions.
{[}6.0 -- 6.9{]} & Thoughtful insights on model reliability and areas
for improvement. {[}7.0 -- 7.9{]} & Critical and structured review
supported with benchmarking or literature comparisons. {[}8.0 --
10.0{]} \\
& & & & & \\
5. Report Presentation (10 marks) & Poorly structured, major
formatting/grammar issues. {[}0 -- 4.9{]} & Basic structure and
formatting, some clarity issues. {[}5.0 -- 5.9{]} & Clear layout with
acceptable writing and referencing. {[}6.0 -- 6.9{]} & Professional
formatting and writing with minimal errors. {[}7.0 -- 7.9{]} & Polished,
professional report, free of errors, in the tone of a graduate engineer.
{[}8.0 -- 10.0{]} \\
\end{longtable}

\end{landscape}

\newpage{}

\section*{References}\label{references-1}
\addcontentsline{toc}{section}{References}

\phantomsection\label{refs}
\begin{CSLReferences}{1}{0}
\bibitem[\citeproctext]{ref-manrique2012ijcre}
Manrique, Yaidelin A., Carlos V. Miguel, Diogo Mendes, Adelio Mendes,
and Luis M. Madeira. 2012. {``Modeling and Simulation of a Packed-Bed
Reactor for Carrying Out the Water-Gas Shift Reaction.''}
\emph{International Journal of Chemical Reactor Engineering} 10 (1).
\url{https://doi.org/10.1515/1542-6580.3105}.

\bibitem[\citeproctext]{ref-mendes2010iecr}
Mendes, Diogo, Vânia Chibante, Adélio Mendes, and Luis M. Madeira. 2010.
{``Determination of the Low-Temperature Water−gas Shift Reaction
Kinetics Using a Cu-Based Catalyst.''} \emph{Industrial \& Engineering
Chemistry Research} 49 (22): 11269--79.
\url{https://doi.org/10.1021/ie101137b}.

\bibitem[\citeproctext]{ref-zhou2023joec}
Zhou, Limin, Yanyan Liu, Shuling Liu, Huanhuan Zhang, Xianli Wu, Ruofan
Shen, Tao Liu, et al. 2023. {``For More and Purer Hydrogen-the Progress
and Challenges in Water Gas Shift Reaction.''} \emph{Journal of Energy
Chemistry} 83 (August): 363--96.
\url{https://doi.org/10.1016/j.jechem.2023.03.055}.

\end{CSLReferences}




\end{document}
